\documentclass[a4paper,10pt]{article}
\usepackage{graphicx} % Required for inserting images
\usepackage[nswissgerman]{babel}
%4 stackanchor  
\usepackage{stackengine}
% define nice looking boxes
\usepackage[many]{tcolorbox}

% a base set, that is then customised
\tcbset {
  base/.style={
    boxrule=0mm,
    leftrule=1mm,
    left=1.75mm,
    arc=0mm, 
    fonttitle=\bfseries, 
    colbacktitle=black!10!white, 
    coltitle=black, 
    toptitle=0.75mm, 
    bottomtitle=0.25mm,
    title={#1}
  }
}
% Mathematical typesetting & symbols
\usepackage{amsthm, mathtools, amssymb} 
\usepackage{marvosym, wasysym}


\allowdisplaybreaks

% Tables
\usepackage{tabularx, multirow}
\usepackage{booktabs}
\renewcommand*{\arraystretch}{2}

% Make enumerations more compact
\usepackage{enumitem}
\setitemize{itemsep=0.5pt}
\setenumerate{itemsep=0.75pt}

% To include sketches & PDFs
\usepackage{graphicx}

% For hyperlinks
\usepackage{hyperref}
\hypersetup{
  colorlinks=true
}
% Math helper stuff
\def\limn{\lim_{n\to \infty}}
\def\limxo{\lim_{x\to 0}}
\def\limxi{\lim_{x\to\infty}}
\def\limxn{\lim_{x\to-\infty}}
\def\sumk{\sum_{k=1}^\infty}
\def\sumn{\sum_{n=0}^\infty}
\def\R{\mathbb{R}}
\def\dx{\text{ d}x}
\usepackage[utf8]{inputenc}

\title{Generelle Information}
\author{Konstantin Lucny}
\date{September 2023}

\begin{document}

\maketitle
\section{Prüfungen}
\begin{itemize}
    \item 2 midterms wenn durchschnitt 4 bestanden endprüfung nicht nötig (darf man dann nicht machen wenn man midterm note will) oder nur endprüfung
    \item \textbf{50\%} der Punkte von den Übungsblättern \textbf{müssen gelöst} sein, damit man bei mid-Terms antreten darf
    \item Mid-terms werden ca. 2. November und 2. Dezember Woche sein (Normalerweise am späteren Nachmittag)
\end{itemize}
\section{Übungsbetrieb}
\begin{itemize}
    \item Jede Woche ein Aufgabenblatt Freitags nach der Vorlesung erscheinen (eine Woche zur Lösung gegeben; Abgabe auf Moodle)
    \item Aufgaben dürfen in klein Gruppen von bis zu 3 Leuten gelöst werden (am besten in der gleichen Übungsgruppe)
    \item Kursinformation auf Moodle
    \item Übungsgruppen werden heute um 12:00
    \item Übungen ab 25.9
    \item Erstes Übungsblatt 22.9
\end{itemize}

\section{Tipps}
\begin{itemize}
    \item Übungsbespiele aus dem Buch ähnlich zu Prüfungsbeispielen
\end{itemize}

\end{document}
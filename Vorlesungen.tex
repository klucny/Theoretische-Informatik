\documentclass[a4paper,10pt]{article}
\usepackage{graphicx} % Required for inserting images
\usepackage[nswissgerman]{babel}
%4 stackanchor  
\usepackage{stackengine}
% define nice looking boxes
\usepackage[many]{tcolorbox}
\usepackage{geometry}
\geometry{margin=1.2in}

% a base set, that is then customised
\tcbset {
  base/.style={
    boxrule=0mm,
    leftrule=1mm,
    left=1.75mm,
    arc=0mm, 
    fonttitle=\bfseries, 
    colbacktitle=black!10!white, 
    coltitle=black, 
    toptitle=0.75mm, 
    bottomtitle=0.25mm,
    title={#1}
  }
}
\definecolor{brandblue}{rgb}{0.34, 0.7, 1}
\newtcolorbox{mainbox}[1]{
  colframe=brandblue, 
  base={#1}
}

\definecolor{orange}{rgb}{1, 0.55, 0.3}
\newtcolorbox{tbox}[1]{
  colframe=orange, 
  base={#1}
}

\definecolor{green}{rgb}{0.294, 0.729, 0.254}
\newtcolorbox{bembox}[1]{
  colframe=green, 
  base={#1}
}

\definecolor{red}{rgb}{0.99, 0.04, 0.99}
\newtcolorbox{tipbox}[1]{
  colframe=red, 
  base={#1}
}

\newtcolorbox{defbox}[1]{
  colframe=black!20!white,
  base={#1}
}
% Mathematical typesetting & symbols
\usepackage{amsthm, mathtools, amssymb} 
\usepackage{marvosym, wasysym}


\allowdisplaybreaks

% Tables
\usepackage{tabularx, multirow}
\usepackage{booktabs}
\renewcommand*{\arraystretch}{2}

% Make enumerations more compact
\usepackage{enumitem}
\setitemize{itemsep=0.5pt}
\setenumerate{itemsep=0.75pt}

% To include sketches & PDFs
\usepackage{graphicx}

% For hyperlinks
\usepackage{hyperref}
\hypersetup{
  colorlinks=true
}
% Math helper stuff
\def\limn{\lim_{n\to \infty}}
\def\limxo{\lim_{x\to 0}}
\def\limxi{\lim_{x\to\infty}}
\def\limxn{\lim_{x\to-\infty}}
\def\sumk{\sum_{k=1}^\infty}
\def\sumn{\sum_{n=0}^\infty}
\def\R{\mathbb{R}}
\def\dx{\text{ d}x}
\usepackage[utf8]{inputenc}

\title{Vorlesungen}
\author{Konstantin Lucny}
\date{HS 2023}

\begin{document}
\maketitle

\section{Vorlesung}
\subsection{Einführung}
$\dfrac{A, A=> B}{B}$
\section{Vorlesung (\today)}
\subsection{Alphabet}
\begin{itemize}
    \item endlich viele Zeichen, womit man womöglich unendlich viele Objekte beschreiben kann (e.g. Alphabet  $\sum=\{0,1\}$ [binär Alphabet])
    \item \textbf{Def. }Wort: Folge von Symbolen aus Alphabet
    \item \textbf{Def.} leeres Wort: $\lambda$ (hat Länge $0$)
    \item \textbf{Def.} Länge: $|aabbbcdd|=8$ und \# Vorkommnis eines bestimmten Symbols $|aabbbcdd|_d=2$
    \item $\sum,...,\sum^*$ die Menge aller Wörter über $\sum$
\end{itemize}
\begin{tipbox}
    {Beispiel: Alphabet}
    $\sum=\{0,1\}$
    \\ $\sum^*=\{y_1,y_2,..,y_k|k\in\mathbb N, y_j \in\{0,1\}  \text{ für } j=1,...,k\}\}$
    $ = \{\lambda, 0, 1, 00, 01, 10, 11,...\}$
\end{tipbox}
\subsubsection{Konkatenation}
$\sum=\{a,b,c\}$ $\sum^*\times\sum^*\rightarrow\sum^*$ 
\\ \textbf{Def.}: $(\sum^*,\cdot)$ (Monuid) (z.B. $abbc\cdot bbac=abbcbbac$)
\textbf{Neutral Element}: $\lambda$
\begin{defbox}
    {Definition: Teilwort}
    \textit{v} heisst Teilwort von \textit{w} $\iff \exists x,y\in\sum^*:w=xvy$ (unformal: \textit{v} ist eine Teilfolge von \textit{w}.
\end{defbox}
\begin{defbox}
    {Definition: Präfix und Suffix}
    \textit{v} heisst ein \underline{Präfix} von \textit{w} $\iff \exists y \in \sum^*:w=vy$
    \textit{y} heisst ein \underline{Suffix} von \textit{w} $\iff \exists v \in \sum^*:w=vy$
\end{defbox}
\textbf{Übersicht Menge}
\begin{itemize}
    \item Menge A
    \item $|A|$ die Kardinalität
    \item $P(A)$ die Potenzmenge von \textit{A}
\end{itemize}
\begin{defbox}
    {Kanonische Ordnung}
    $\sum=\{s_1,s_2,...,s_m\}$; $s_1<s_2<...<s_m$
\end{defbox}
\begin{defbox}
    {Ordnung in $\sum^*$}
    $u<v\iff |u|<|v| \vee |u|=|v|\wedge =u=xs_i\cdot u'\wedge v=xs_j\cdot v'\wedge s_i<s_j$
\end{defbox}
\subsubsection{Sprache}
\textbf{Def.} $\sum,\sum^*$ eine \underline{Sprache} L über $\sum^*$ .. $L\subseteq\sum^*$\\[3pt]
\textbf{Komplement}: $L^C=\sum^*-L$\\[3pt]
\textbf{Mengen Operationen}: $L_1 \cup L_2, L_1\cap L_2, L_1-L_2$ \\[3pt]
\textbf{Leere Sprache}: $L_{\emptyset}=\emptyset$; $L_\lambda=\{\lambda\}$\\[3pt]
$L_1\cdot L_2 = L_2L_1=\{vw|v\in L_1 \wedge w\in L_2\}$\\[3pt]
\textbf{Bsp.} $\{a^nb^n|n\in\mathbb N\}\cdot\{c^mb^{2m}|m\in\mathbb N\}=\{a^nb^nc^md^{2m}|n\in\mathbb N, m\in \mathbb N\}$
\begin{bembox}
    {Konkatenation mit div. Mengen}
    \begin{itemize}
        \item     $L\cdot\emptyset=\emptyset$ 
        \item     $L\cdot L_\lambda=L$ 
        \item     $L^0=L_\lambda=\{\lambda\}$
        \item    $L^1=L$
        \item    $L^{i+1}=L\cdot L^i$
        \item $L^+=\displaystyle\bigcup_{i\in\mathbb N -{0}} L^i=L\cdot L^*$
        \item $L*=\displaystyle\bigcup_{i\in \mathbb N}L^i$
    \end{itemize}
\end{bembox}
\begin{tbox}
    {Assoziativität Konkatenation}
    $L_1\cdot L_2\wedge L_1\cdot L_3= L_1( $\\
    Beiweis in Notizen \today
\end{tbox}
\subsection{Homomorphismus}
$\sum_1, \sum_2$ \\[4pt]
$h:\sum_1^*\rightarrow\sum_2^*$\\[3pt]
$(i)\ h(\lambda)=\lambda$\\[3pt]
$(ii)\  h(uv)=h(u)\cdoth(v);\forall u,v\in\sum^*$\\[3pt]
$h(a_1\cdot...\cdot a_n)=h(a_1)\cdot ... \cdot h(a_n); (a_i\in\sum)$\\[3pt]

$A:\underbrace{\textstyle{\sum_1^*}}_{x}\rightarrow\underbrace{\textstyle\sum_2^*}_{A(x)}$\\
\textbf{Def.} Äquivalenz: A äquivalent zu B: $\forall x .\in \sum_1^*: A(x)=B(x)$

\pagebreakcc
\subsection{Entscheidungsproblem ($\sum, \Gamma$)}
$L\subseteq\sum^*$\\[4pt]
\textit{A} löst $(\sum,\Gamma)$\\[4pt]
$\forall x \in \sum^*: A(x)
\begin{cases}
    1 & \text{falls } x \in \Gamma\\
    0 & \text{falls } x \notin \Gamma
\end{cases}$

\end{document}
